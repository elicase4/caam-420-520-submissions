\documentclass[11pt]{article}

\usepackage[lmargin=0.75in, rmargin=1in, tmargin=1.25in, bmargin=1in]{geometry}
\usepackage[english]{babel}
\usepackage[utf8]{inputenc}

\usepackage{amsfonts}
\usepackage{amsmath}
\usepackage{bm}
\usepackage{booktabs}
\usepackage[labelsep=period]{caption}
\usepackage{enumitem}
\usepackage{fancyhdr}
\usepackage{graphicx}
\usepackage{lastpage}
\usepackage{listings}
\usepackage[svgnames]{xcolor}

% For formatting C++ code listings ---------------------------
\lstdefinestyle{customCpp}{
    language=C++,
    keywordstyle=\color{RoyalBlue},
    basicstyle=\footnotesize\ttfamily,
    commentstyle=\color{Green}\ttfamily,
    rulecolor=\color{black},
    numbers=left,
    numberstyle=\tiny\color{gray},
    stepnumber=1,
    numbersep=8pt,
    showstringspaces=false,
    breaklines=true,
    frame = tb,
    belowcaptionskip=5pt,
    belowskip=3em,
    gobble=10,
}

% For turning enumerated lists into Problem titles --------------
\renewcommand{\labelenumi}{\textbf{Problem \arabic{enumi}}}
\renewcommand{\labelenumii}{(\alph{enumii})}

% Enumerated list indents:
% Problems:    [leftmargin=0.9in]
% Subproblems: [leftmargin=0.3in]


% Document Details ----------------------------------------------
\author{Eli Case}
\title{CAAM 420/520 -- Homework 3}
\date{March, 10, 2023}
\makeatletter


% Setup headers -------------------------------------------------
\pagestyle{fancy}
\fancyhf{} % Clear the headers and footers
\lhead{\@author}
\chead{\@title}
\rhead{\@date}
\cfoot{Page \thepage\ of \pageref{LastPage}}
\setlength{\headheight}{15pt}
\setlength{\headsep}{20pt}

\fancypagestyle{plain}{
	\fancyhf{}
	\setlength{\headheight}{15pt}
	\setlength{\headsep}{0pt}
	\renewcommand{\headrulewidth}{0pt}
	\cfoot{\vspace{2mm}Page \thepage\ of \pageref{LastPage}}
}


\begin{document}
\flushleft
\thispagestyle{plain}
To: Christina Taylor

From: \@author

Date: \@date

Subject: \@title

\makeatother
\medskip
\hrule
\medskip

\begin{enumerate}[leftmargin=0.9in]
\item % Problem 1

   \begin{enumerate}[leftmargin=0.3in]
       \item
           Nodes at a given level in the function tree can be processed at the same time. For example, we can begin traversing the function tree in Figure 1 at level 0. Once we reach the child nodes of level 0 at level 1, we can process the child nodes of level 1 simultaneously, as processing each child node is an independent operation.
       \item
           In evaluating the function tree, synchronization is required after the subtree for a given operator node has been processed. For example, Before evaluating the multiplication node at level 1, we must evaluate the subtree for the addition node at level 2 fist, for the function tree given in Figure 1.
       \setcounter{enumii}{3}
       \item
           To find the parallel time complexity of evaluating the sun function using a function tree, we can make start by making the following observations. 
           \begin{enumerate}
               \item If the number of nodes, $n$, in a given layer, $L$, is less than the number of total threads, $N_T$, then the time complexity to process that layer is $\mathcal{O}(1)$.
               \item If the number of nodes, $n$, in a given layer, $L$, is larger than or equal to the number of total threads, $N_T$, then the time complexity to process that layer is $\mathcal{O}(\frac{2^L}{N_T})$.
           \end{enumerate}
           Now, given that the number of threads is a power of $2$, as $N_T = 2^T$, and that $L=T$, when the number of nodes in a given layer equal the number of threads, we have the following equation for the time complexity of processing a given layer in parallel, {\fontfamily{cmtt}\selectfont time\_layer}, in the function tree.
           \begin{equation} \label{eq:1}
               \text{{\fontfamily{cmtt}\selectfont time\_layer}}  = \begin{cases}
                   \mathcal{O}(1) & \text{$L < T$} \\
                   \mathcal{O}(\frac{2^L}{N_T}) & \text{$L \geq T$}
               \end{cases}
           \end{equation}
           Subsequently, we can compute the total time complexity of processing the function tree, {\fontfamily{cmtt}\selectfont time\_tree}, by adding the time complexity for each layer, as given in Equation \ref{eq:1}.
           \begin{equation} \label{eq:2}
               \text{{\fontfamily{cmtt}\selectfont time\_tree}} = \mathcal{O}\bigg(\sum_{i = 0}^{T - 1} 1 + \frac{1}{N_T}\sum_{i = T}^{L_{\text{max}}} 2^i \bigg)
           \end{equation}
           We can rewrite Equation \ref{eq:2}, in terms of the variables of interest, $N_T$ and $n$, with the following formulas.
           \begin{align*}
               T &= log_2(N_T) \\
               n &= 2m - 1 \\ 
               L_{max} &= log_2(m)
           \end{align*}
           Rewriting Equation \ref{eq:2} and simplifying, we have the following equation for {\fontfamily{cmtt}\selectfont time\_tree}.
           \begin{equation} \label{eq:3}
               \text{{\fontfamily{cmtt}\selectfont time\_tree}} = \mathcal{O}\bigg(log_2(N_T) +  \frac{1}{N_T}\sum_{i = log_2(N_T)}^{log_2(\frac{n + 1}{2})} 2^i \bigg)
           \end{equation}
           Now, we can simplify Equation \ref{eq:3} by rewriting the finite sum of the geometric series. We have the following formula for the sum of a finite geometric series, $S_n$ given by the following formula.
           \begin{equation} \label{eq:4}
               S_n = \sum_{i = 0}^{n - 1} r^i = \frac{r^n - 1}{r - 1}, \quad \text{$r > 1$}
           \end{equation}
           We would like to rewrite the following finite sum geometric of a geometric series.
           \begin{equation} \label{eq:5}
               \sum_{i = log_2(N_T)}^{log_2(\frac{n + 1}{2})} 2^i
           \end{equation}
           In order to apply Equation \ref{eq:4}, we split the finite sum of a geometric series given in Equation \ref{eq:5} as follows.
           \begin{equation} \label{eq:6}
               \sum_{i = 0}^{log_2(\frac{n + 1}{2})} 2^i - \sum_{i = 0}^{log_2(N_T) - 1} 2^i 
           \end{equation}
           Now, using Equation \ref{eq:4}, we rewrite Equation \ref{eq:6}.
           \begin{align*}
               \frac{2^{log_2(\frac{n + 1}{2}) + 1} - 1}{2 - 1} &- \frac{2^{log_2(N_T)} - 1}{2 - 1} \\
               2^{log_2(\frac{n + 1}{2}) + log_2(2)} &- 2^{log_2(N_T)} \\
               2^{log_2(n + 1)} &- 2^{log_2(N_T)} \\
               n + 1 &- N_T 
           \end{align*}
           Thus, we can now simplify Equation \ref{eq:3}.
           \begin{align*}
               \text{{\fontfamily{cmtt}\selectfont time\_tree}} &= \mathcal{O}\bigg(log_2(N_T) + \frac{1}{N_T}\big(n + 1 - N_T \big) \bigg) \\
               \text{{\fontfamily{cmtt}\selectfont time\_tree}} &= \mathcal{O}\bigg(log_2(N_T) + \frac{n + 1}{N_T} - 1 \bigg)
           \end{align*}
           Finally, we obtain the time complexity for processing the function tree in parallel, as shown below.
           \begin{equation} \label{eq:7}
               \mathcal{O}\bigg(log_2(N_T) + \frac{n + 1}{N_T} - 1 \bigg)
           \end{equation}
       \item
           We were given that the time complexity for computing the sum with a parallel for loop was as follows.
           \begin{equation*}
               \mathcal{O}\bigg( \frac{m}{N_T} \bigg)
           \end{equation*}
           Rewriting the above formula in terms of $n$, we obtain the following time complexity.
           \begin{equation} \label{eq:8}
               \mathcal{O}\bigg( \frac{n + 1}{2N_T} \bigg)
           \end{equation}
           We can determine when the parallel for loop implementation is slower than the function tree implementation in parallel by solving the following inequality.
           \begin{equation} \label{eq:9}
               \frac{n + 1}{2N_T} > log_2(N_T) + \frac{n + 1}{N_T} - 1
           \end{equation}
           Now, we simplify the inequality given in Equation \ref{eq:9} as follows.
           \begin{equation} \label{eq:10}
               log_2(N_T) + \frac{n+1}{2N_T} < 1
           \end{equation}
           If we consider the smallest function tree, $n=3$, and only one thread, $N_T = 1$, it can be seen that Inequality \ref{eq:10} is not satisfied. As a result, the function tree implementation of evaluating the sum function is always slower than the parallel for loop implementation, in both serial and parallel.  

   \end{enumerate} % End of Problem 1 subpoints

\end{enumerate} % End of Problems

\end{document}
